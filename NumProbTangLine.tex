\documentclass[ebook,12pt,oneside,openany]{memoir} %ebook format
\usepackage[T1]{fontenc} %Font adjustment
\usepackage{MnSymbol, fullpage, adjustbox, fancyhdr, array} %Symbols, page format, \maxsizebox, custom page numbering, custom width (centered) columns
\renewcommand{\headrulewidth}{0pt} %FANCYHDR Eliminates line from header
\cfoot{\thepage$^{\prime}$} %FANCYHDR Custom page numbering
\newcommand{\PreserveBackslash}[1]{\let\temp=\\#1\let\\=\temp} %ARRAY
\newcolumntype{C}[1]{>{\PreserveBackslash\centering}p{#1}} %ARRAY
\setlength{\parindent}{0pt} %No indent
%To use python within LaTeX; pyfuture allows me to use Python 3.9.  
\usepackage{pythontex}[pyfuture=all]
%NOTE: My command propmt directory has preloaded path to the folder which this document.tex resides.
%Compiling my LaTeX document adjusts all LaTeX info, and logs the Python commands I want to use > From the command propmt, I run ``python pythontex.py DOCUMENT NAME'' in order to prepare the Python code for my LaTeX > Compiling my LaTeX documents utilizes the prepared Python code for use in my PDF!
\begin{document}
\begin{sympycode}

################################################
#QUICK START:  Here are the only options you'll need to customize the generated document! 


#How many problems do you want?  (Set NumProb > 4)

NumProb=104


#For anew problem set, add a hashtag to the Version Number, below!
#Then recompile LaTeX > Python > LaTeX

#Version Number:

#1 #2 

#END OF QUICK START
################################################





print(r'\title{{\Huge \textbf{', NumProb, r'}} \\ \textbf{Tangent Line Exercises}\\ \large\textit{\textbf{(With SOLUTIONS!)}}}')
print(r'\author{Jonathan D.\ Doane\\ Calc.\ 1}')

\end{sympycode}

%Here is my phython compile command for my command prompt; see \usepackage{pythontex}.
%I will put this command throughout my document for convenience.

%python pythontex.py NumProbTangLine


%Here is a list of lengths I can adjust.  Putting these commands in my python code allows me to re-compile my document without the command propmt!

\def\OutColWidthSml{.3in}
\def\OutColWidthSml{.28in}
\def\OutColWidthMed{.7in}
\def\InColWidthMed{.68in}
\def\OutColWidthBig{1.1in}
\def\InColWidthBig{1in}
\def\ColHeight{15pt}


%Print my title

\maketitle


%Finish my cover with a preview.  

\vfill

{\large \textbf{Preview}}


%In order to have material TO preview, I must run some Python code to MAKE some material!
%Sympy in order to have sin, pi, and sin(pi)=0; etc.

\begin{sympycode}











#Can get random.choice element from given list.

import random


#Allow x to be used by sympy as a variable.

var('x')


#My functions; I want my exercises to be relatively simple.  We'll add (x+y)**z to it, soon!

funcs = [sin(x), cos(x), tan(x)]


#To store our results

F=[]
D=[]
A=[]
M=[]
Y=[]
T=[]


#Let's make some problems!  

for func in range(NumProb):


#Store the index func as i

	i=func 


#We're adding two (x+y)**z so we don't get an overwhelming number of trig problems! 

	y=int(random.choice(range(-10,10,1)))
	yy=int(random.choice(range(-10,10,1)))
	z=int(random.choice([-2,-1,1,2]))
	zz=int(random.choice([-2,-1,1,2]))
	funcs.append((x+y)**z)
	funcs.append((x+yy)**zz)  


#We'll make one hundred problems randomly, and four additional specific problems (to display my four types of functions in the Preview on my cover)!

	func=random.choice(funcs)
	while i==(NumProb-4) and func!=cos(x):
			func=random.choice(funcs)
	while i==(NumProb-3) and func!=(x+y)**z:
		func=random.choice(funcs)
	while i==(NumProb-2) and func!=sin(x):
		func=random.choice(funcs)
	while i==(NumProb-1) and func!=tan(x):
		func=random.choice(funcs)


#We want nice numbers for our trig functions; e.g. tan(pi/2) isn't defined, and sin(1) would be terrible!

	if func==tan(x):
		func=y*func


#Sympy wants the derivative of tan(x) to be tan(x)**2+1; let's fix that!

		dfunc=y*(sec(x)**2)
		a=random.choice([-2*pi, -pi,0,pi,2*pi, pi/3, (2*pi)/3, pi/6, -pi/3, -pi/6, (-2*pi)/3])


#We want to record that the numbers involved with trig functions are typically NOT RATIONAL, e.g. sin(pi/3)=sqrt(3)/2.

		RatValid=False
	elif func==sin(x) or func==cos(x):
		func=zz*func
		dfunc=diff(func,x)
		a=random.choice([-2*pi, (-3*pi)/4, (-pi)/2, (-pi)/4, -pi,0,pi,2*pi, (3*pi)/4, (pi)/2, (pi)/4, pi/3, -pi/3, pi/6, -pi/6, (-2*pi)/3, (2*pi)/3])
		RatValid=False
	else:
		dfunc=diff(func,x)
		if func==(x+y)**z:


#We want answers with short answers; here is some assurance for that!

			a=int(-1*random.choice(range(y-2,y+2,1)))
			while a==int(-1*y):
				a=int(-1*random.choice(range(y-2,y+2,1)))
		else:
			a=int(-1*random.choice(range(yy-2,yy+2,1)))
			while a==int(-1*yy):
				a=int(-1*random.choice(range(yy-2,yy+2,1)))


#We want to record that the numbers involved with this function ARE RATIONAL!

		RatValid=True


#Now that we've built our material, let's record it!

	A.append(a)
	F.append(func)
	D.append(dfunc)


#Look, we even get to use our RATIONAL truth values!

	M.append(nsimplify(dfunc.evalf(subs={x: a}), rational=RatValid))
	Y.append(nsimplify(func.evalf(subs={x: a}), rational=RatValid))
	T.append(nsimplify(Y[i]+M[i]*(x-a), rational=RatValid))


#We remove our two functions so we don't accidentally increase their number!

	funcs.remove((x+y)**z)
	funcs.remove((x+yy)**zz)

\end{sympycode}

%python pythontex.py NumProbTangLine


%Okay, NOW we can build our preview!

\begin{sympycode}

#Using r'MyStuff' allows us to use latex commands; awesome!  This is great, because I was having a tough time trying to get Python to BOLDen my item numbers; I can have LaTeX do it for me, instead!

print(r'\begin{enumerate}')


#Let's BOLDen our item numbers, and start our count at 61 for our preview!

print(r'\renewcommand\labelenumi{\bfseries\theenumi.}')
print(r'\setcounter{enumi}{',(NumProb-4),r'}')


#PYTHON will iterate for us, LATEX will print for us!!!

for n in range((NumProb-4),NumProb,1):
	print(r'\item ')
	print(r' \begin{tabular}{C{\OutColWidthMed}C{\OutColWidthMed}C{\OutColWidthSml}}')


#Be careful to use mathmode around the latex() command.

	print(r'\maxsizebox{\InColWidthMed}{!}{$', latex(F[n]), r'$} & at  $x_1=$ & \maxsizebox{\OutColWidthSml}{!}{$',  latex(A[n]), r'$} \\')
	print(r'\end{tabular}')

print(r'\end{enumerate}')


#Let's add their solutions in a table with custom column widths.

print(r' \begin{tabular}{|C{\OutColWidthSml}|C{\OutColWidthMed}|C{\OutColWidthSml}|C{\OutColWidthSml}|C{\OutColWidthMed}|C{\OutColWidthSml}|C{\OutColWidthBig}|} ')
print(r' \hline ')

#Using \prime helps us avoid using ' and causing an error!

print(r'    \textbf{n.}    &     $f(x)$      &    $x_1$       &     $y_1$       &     $f^{\prime}(x)$     &     $m$     &  \vphantom{\resizebox{!}{\ColHeight}{$|$}}    tangent line  \vphantom{\resizebox{!}{\ColHeight}{$|$}}  \\ ')
print(r' \hline ')
print(r' \hline ')


for n in range((NumProb-4),NumProb,1):

	if n!=0 and n % 20 ==19:
		print(r'\maxsizebox{\OutColWidthSml}{!}{   \textbf{', n+1, r'\hspace*{-.07in}.}   } & \maxsizebox{\InColWidthMed}{!}{   $', latex(F[n]), r'$   } & \maxsizebox{\OutColWidthSml}{!}{  $', latex(A[n]), r'$  } & \maxsizebox{\OutColWidthSml}{!}{  $', latex(Y[n]), r'$  } & \maxsizebox{\InColWidthMed}{!}{   $', latex(D[n]), r'$  } & \maxsizebox{\OutColWidthSml}{!}{  $', latex(M[n]), r'$   } & \vphantom{\resizebox{!}{\ColHeight}{$|$}} \maxsizebox{\InColWidthBig}{!}{     $y=', latex(T[n]), r'$   } \vphantom{\resizebox{!}{\ColHeight}{$|$}} \\ ')
		print(r' \hline ')
		print(r'\end{tabular} ')
		print(r'\newpage ')
		print(r' \begin{tabular}{|C{\OutColWidthSml}|C{\OutColWidthMed}|C{\OutColWidthSml}|C{\OutColWidthSml}|C{\OutColWidthMed}|C{\OutColWidthSml}|C{\OutColWidthBig}|} ')
		print(r' \hline ')
		print(r'    \textbf{n.}    &     $f(x)$      &    $x_1$       &    $y_1$       &     $f^{\prime}(x)$     &     $m$     &  \vphantom{\resizebox{!}{\ColHeight}{$|$}}  tangent line  \vphantom{\resizebox{!}{\ColHeight}{$|$}}  \\ ')
		print(r' \hline ')
		print(r' \hline ')

	else:
		print(r'\maxsizebox{\OutColWidthSml}{!}{   \textbf{', n+1, r'\hspace*{-.07in}.}    } &  \maxsizebox{\InColWidthMed}{!}{   $', latex(F[n]), r'$    }  & \maxsizebox{\OutColWidthSml}{!}{   $', latex(A[n]), r'$    } &  \maxsizebox{\OutColWidthSml}{!}{   $', latex(Y[n]), r'$  } & \maxsizebox{\InColWidthMed}{!}{   $', latex(D[n]), r'$   }  &  \maxsizebox{\OutColWidthSml}{!}{   $', latex(M[n]), r'$   } & \vphantom{\resizebox{!}{\ColHeight}{$|$}} \maxsizebox{\InColWidthBig}{!}{  $y=', latex(T[n]), r'$   } \vphantom{\resizebox{!}{\ColHeight}{$|$}} \\ ')
		print(r' \hline ')


print(r'\end{tabular} ')

\end{sympycode}

%python pythontex.py NumProbTangLine


%Eliminate the page number on this cover page!

\thispagestyle{empty}


%%%%%%%%%%%%%%%%%%%End Cover
%%%%%%%%%%%%%%%%%%%%%%%%%%%%%%%%%%%%%%%%%%%%%%%

\newpage










%Let's start with a fresh page counter!

\clearpage
\setcounter{page}{1}


%We want the directions and first twenty problems to fit on this page.

\vspace*{-.4in}

\resizebox{\linewidth}{!}{Find the equation of the tangent line for each of the following:}

\vspace*{-.1in}


%Here we go; let's make all 104 problems!

\begin{sympycode}

print(r'\begin{enumerate}')
print(r'\renewcommand\labelenumi{\bfseries\theenumi.}')
print(r'\setcounter{enumi}{0}')

for n in range(NumProb):
	print(r'\item ')
	print(r' \begin{tabular}{C{\OutColWidthMed}C{\OutColWidthMed}C{\OutColWidthSml}}')
	print(r'\maxsizebox{\InColWidthMed}{!}{$', latex(F[n]), r'$} & at  $x_1=$ & \maxsizebox{\OutColWidthSml}{!}{$',  latex(A[n]), r'$} \\')
	print(r'\end{tabular}')


#We can even force page breaks after each set of twenty problems!

	if n % 20 ==19:
		print(r'\newpage')
	else:
		pass

print(r'\end{enumerate}')

\end{sympycode}

%python pythontex.py NumProbTangLine


%%%%%%%%%%%%%%%%%%%End Page 4
%%%%%%%%%%%%%%%%%%%%%%%%%%%%%%%%%%%%%%%%%%%%%%%

\newpage










%Let's start with a fresh page counter, AND custom label the counter! \usepackage{fancyhdr}

\clearpage
\setcounter{page}{1}
\pagestyle{fancy}


{\large \textbf{Solutions}}

\bigskip

\begin{sympycode}


print(r' \begin{tabular}{|C{\OutColWidthSml}|C{\OutColWidthMed}|C{\OutColWidthSml}|C{\OutColWidthSml}|C{\OutColWidthMed}|C{\OutColWidthSml}|C{\OutColWidthBig}|} ')
print(r' \hline ')
print(r'    \textbf{n.}    &     $f(x)$      &    $x_1$       &     $y_1$       &     $f^{\prime}(x)$     &     $m$     &   \vphantom{\resizebox{!}{\ColHeight}{$|$}}   tangent line \vphantom{\resizebox{!}{\ColHeight}{$|$}}   \\ ')
print(r' \hline ')
print(r' \hline ')

for n in range(NumProb):
	if n!=0 and n % 20 ==19:
		print(r'\maxsizebox{\OutColWidthSml}{!}{   \textbf{', n+1, r'\hspace*{-.07in}.}   } & \maxsizebox{\InColWidthMed}{!}{    $', latex(F[n]), r'$        } & \maxsizebox{\OutColWidthSml}{!}{   $', latex(A[n]), r'$       } & \maxsizebox{\OutColWidthSml}{!}{   $', latex(Y[n]), r'$       } & \maxsizebox{\InColWidthMed}{!}{    $', latex(D[n]), r'$       } & \maxsizebox{\OutColWidthSml}{!}{   $', latex(M[n]), r'$      } & \vphantom{\resizebox{!}{\ColHeight}{$|$}} \maxsizebox{\InColWidthBig}{!}{      $y=', latex(T[n]), r'$   } \vphantom{\resizebox{!}{\ColHeight}{$|$}} \\ ')
		print(r' \hline ')
		print(r'\end{tabular} ')
		print(r'\newpage ')
		print(r' \begin{tabular}{|C{\OutColWidthSml}|C{\OutColWidthMed}|C{\OutColWidthSml}|C{\OutColWidthSml}|C{\OutColWidthMed}|C{\OutColWidthSml}|C{\OutColWidthBig}|} ')
		print(r' \hline ')
		print(r'    \textbf{n.}    &     $f(x)$      &    $x_1$       &    $y_1$       &     $f^{\prime}(x)$     &     $m$     &  \vphantom{\resizebox{!}{\ColHeight}{$|$}}  tangent line  \vphantom{\resizebox{!}{\ColHeight}{$|$}}  \\ ')
		print(r' \hline ')
		print(r' \hline ')
	else:
		print(r'\maxsizebox{\OutColWidthSml}{!}{   \textbf{', n+1, r'\hspace*{-.07in}.}    } &  \maxsizebox{\InColWidthMed}{!}{    $', latex(F[n]), r'$        }  & \maxsizebox{\OutColWidthSml}{!}{   $', latex(A[n]), r'$       } &  \maxsizebox{\OutColWidthSml}{!}{   $', latex(Y[n]), r'$       } & \maxsizebox{\InColWidthMed}{!}{     $', latex(D[n]), r'$      }  &  \maxsizebox{\OutColWidthSml}{!}{   $', latex(M[n]), r'$      } & \vphantom{\resizebox{!}{\ColHeight}{$|$}} \maxsizebox{\InColWidthBig}{!}{      $y=', latex(T[n]), r'$   } \vphantom{\resizebox{!}{\ColHeight}{$|$}} \\ ')
		print(r' \hline ')

print(r'\end{tabular} ')

\end{sympycode}

%python pythontex.py NumProbTangLine


%%%%%%%%%%%%%%%%%%%End Page 4'
%%%%%%%%%%%%%%%%%%%%%%%%%%%%%%%%%%%%%%%%%%%%%%%

\end{document}